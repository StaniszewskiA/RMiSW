\documentclass[12pt,a4paper,table]{article}

\usepackage[a4paper,
            tmargin=2cm,
            bmargin=2cm,
            lmargin=2cm,
            rmargin=2cm,
            bindingoffset=0cm]{geometry}

\usepackage{lmodern}
\usepackage[T1]{polski}
\usepackage[utf8]{inputenc}
\usepackage{tocloft}
\usepackage{hyperref}
\usepackage{amsmath}
\usepackage{listings}
\usepackage{graphicx}
\usepackage{subfig}
\usepackage{float}
\usepackage{booktabs}

\hypersetup{
    colorlinks,
    citecolor=black,
    filecolor=black,
    linkcolor=black,
    urlcolor=black
}

\newtheorem{definition}{Def}


\begin{document}
    \title {
        Rachunek Macierzowy i Statystyka Wielowymiarowa \\
        Sprawozdanie 2 \\
        Eliminacja Gaussa i LU Faktoryzacja

    }

    \author{
        Adam Staniszewski \\
        Przemysław Węglik
    }

    \date{\today}

    \maketitle

    \tableofcontents
    \newpage

    \section{Eliminacja Gaussa}

    \subsection{Algorytm eliminacji Gaussa bez pivotingu}
    \subsubsection{Opis algorytmu}
    Algorytm eliminacji Gaussa polega na przekształceniu macierzy do postaci górnotrójkątnej poprzez iteracyjne zerowanie elementów pod przekątną w kolejnych kolumnach. W każdym wierszu jest wybierany element główny, w wariancie bez pivotingu jest to pierwszy element niezerowy. Na jego podstawie wykonywane są przekształcenia, które mają wyzerować pozostałe elementy w kolumnie. 

    Algorytm składa sięz dwóch części. Pierwsza zaczyna się od podzielenia wartości w pierwszym rzędzie macierzy \textbf{A} i pierwszego elementu wektor \textbf{b}, tak aby wartość pierwszego elementi na przekątnej \textbf{A} wyniosła 1. Kolejne wiersze są redukowane o wartość wyliczoną na podstawie pierwszych elementów poprzedniego i aktualnego wiersza. 
    W drugiej fazie algorytm iteruje od ostatniego wiersza macierzy i odejmuje od kolejnych wierszy odpowiednio przeskalowane poprzednie wiersza, tak aby wyzerować wartości pod przekątną.

    \subsubsection{Pseudokod}

    \begin{lstlisting}[]
    function gauss_elimination(A, b):
        n = length(A)
        
        for i from 0 to n - 1:
            divisor = A[i, i]
            A[i] /= divisor
            b[i] /= divisor
            
            for j from i + 1 to n - 1:
                multiplier = A[j, i]
                A[j] -= multiplier * A[i]
                b[j] -= multiplier * b[i]
        
        for i from n - 1 to 0 step -1:
            for j from 0 to i - 1:
                multiplier = A[j, i]
                A[j] -= multiplier * A[i]
                b[j] -= multiplier * b[i]
        
        return b

    \end{lstlisting}

    \subsubsection{Implementacja algorytmu w Pythonie}
    \begin{lstlisting}[language=Python]
    def gauss_elimination(A, b):
        n = len(A)
        
        for i in range(n):
            divisor = A[i, i]
            A[i] /= divisor
            b[i] /= divisor
            
            for j in range(i + 1, n):
                multiplier = A[j, i]
                A[j] -= multiplier * A[i]
                b[j] -= multiplier * b[i]
    
        for i in range(n - 1, -1, -1):
            for j in range(i):
                multiplier = A[j, i]
                A[j] -= multiplier * A[i]
                b[j] -= multiplier * b[i]
        
        return b

    \end{lstlisting}

    \subsubsection{Analiza złożoności}
    Algorytm wykonuje dzielenie dla wszytkich wierszy macierzy kwadratowej \textbf{A} ($n-i*n-i$) i wszystkich elementów wektora \textbf{b} ($n-i$). Do wszystkich elementów \textbf{A} i \textbf{b} musi zostać dodany współczynnik, do którego wyliczenia jest potrzeba operacja mnożenia. Mamy zatem $(n-1)^2$ operacji mnożenia i dodawania dla A oraz $n-1$ dla b. Sumarycznie mamy zatem:
    \begin{equation}
        \text{Operacje mnożenia} = \sum_{i=1}^{n-1} ((n - i)^2 + 2(n - i))  
    \end{equation}
    \begin{equation}
        \text{Operacje dodawania} = \sum_{i=1}^{n-1} ((n - i)^2 + (n - i))  
    \end{equation}

    Wiedząc, że:
    \begin{equation}
        \sum_{k=1}^{n} k = \frac{n(n + 1)}{2}
    \end{equation}

    oraz:

    \begin{equation}
        \sum_{k=1}^{n} k^2 = \frac{n(n + 1)(2n + 1)}{6}
    \end{equation}

    mamy:

    \begin{equation}
        \text{Operacje mnożenia} = \frac{2n^3 + 3n^2 - 5n}{6}
    \end{equation}
    \begin{equation}
        \text{Operacje dodawania} = \frac{n^3 - n}{3} 
    \end{equation}

    sumarycznie mamy:
    
    \begin{equation}
        \text{Złożoność} = \frac{2n^3}{3} \text{+} \frac{3n^2}{2} \text{-} \frac{7n}{6}
    \end{equation}

    Rząd złożoności wynosi zatem $O(n^3)$
    
    \subsection{Algorytm eliminacji Gaussa z pivotingiem}    
    \subsubsection{Opis algorytmu}
    Zastosowanie \textbf{pivotingu} w algorytmie eleminacji Gaussa ma na celu zredukowanie powstających błędów numercznych. W tym wariancie elementem głównym dla każdego wiersza jest elementu o największej wartości bezwględnej. Jeśli ten element nie znajduje się na przekątnej, to zawierająca go kolumna zostanie przeniesiona w odpowiednie miejsce.
    \subsubsection{Pseudokod}
    
    \begin{lstlisting}[]
    function gauss_elimination_pivoting(A, b):
    n = size(A, 1)
    pivots = []

    for i = 1 to n do:
        pivot_row = max(abs(A[i:, i])) + i - 1
        if pivot_row != i then:
            pivots.append((i, pivot_row))
            swap rows A[i] and A[pivot_row]
            swap elements b[i] and b[pivot_row]

        divisor = A[i, i]
        A[i] /= divisor
        b[i] /= divisor

        for j = i + 1 to n do:
            multiplier = A[j, i]
            A[j] -= multiplier * A[i]
            b[j] -= multiplier * b[i]

    for i = n downto 1 do:
        for j = i - 1 downto 1 do:
            multiplier = A[j, i]
            A[j] -= multiplier * A[i]
            b[j] -= multiplier * b[i]

    for (i, pivot_row) in reversed(pivots) do:
        swap rows A[i] and A[pivot_row]
        swap elements b[i] and b[pivot_row]

    return b
    \end{lstlisting}

    \subsubsection{Implementacja algorytmu w Pythonie}
    \begin{lstlisting}[language=Python]
    def gauss_elimination_pivoting(A, b):
        n = len(A)
        pivots = []
        
        for i in range(n):
            pivot_row = np.argmax(np.abs(A[i:, i])) + i
            if pivot_row != i:
                pivots.append((i, pivot_row))
                A[[i, pivot_row]] = A[[pivot_row, i]]
                b[[i, pivot_row]] = b[[pivot_row, i]]
            
            divisor = A[i, i]
            A[i] /= divisor
            b[i] /= divisor
            
            for j in range(i + 1, n):
                multiplier = A[j, i]
                A[j] -= multiplier * A[i]
                b[j] -= multiplier * b[i]
    
        for i in range(n - 1, -1, -1):
            for j in range(i):
                multiplier = A[j, i]
                A[j] -= multiplier * A[i]
                b[j] -= multiplier * b[i]
    
        for i, pivot_row in pivots[::-1]: 
            A[[i, pivot_row]] = A[[pivot_row, i]]
            b[[i, pivot_row]] = b[[pivot_row, i]]
    
        return b

    \end{lstlisting}
    \subsubsection{Analiza złożoności}
    Wariant z pivotingiem posiada dwie modyfikacje dotyczące złożoności obliczeniowej. Po pierwsze, szukamy elementu głównego, na co potrzebujemy maksymalnie \( n \) operacji. Po drugie, kolejność rzędów musi zostać zamieniona, aby główny element zawsze był pierwszym elementem niezerowym. Zamiana jest operacją o stałej złożoności. Rząd złożoności obliczeniowej nadal wynosi więc \( O(n^3) \).

    
    \subsubsection{Porównanie z Numpy}

    \begin{lstlisting}[language=Python]
        # gauss elimination
        [-1.22364004  1.2599961   2.48610013 -0.76265502]
        # gauss elimination with pivoting
        [-1.22364004  1.2599961   2.48610013 -0.76265502]
        # numpy.linalg.solve
        [-1.22364004  1.2599961   2.48610013 -0.76265502]

        # checking using numpy.allclose
        Are the results close: True
        Are the results close: True

    \end{lstlisting}
    
    Algorytm zwraca identyczne wyniki równania jak funkcji 
    biblioteczna. Jedyne czego można by się dosuzkiwać to różna dokładność numeryczna,
    ale to róœnież zweryfikowano i wyniki są równe z dokładnością $10^{-8}$

    \section{LU Faktoryzacja}

    \subsection{Algorytm LU faktoryzacji bez pivotingu}
    \subsubsection{Opis algorytmu}
    Faktoryzacja LU polega na dekompozycji macierzy kwadratowej na iloczyn dwóch macierzy trójkątnych - dolnej (L) i górnej (U). Macierz \textbf{L} jest inicjowana jako jednostokowa, a \textbf{U} jako kopia maciery \textbf{A}. Algorytm modyfikuje macierz \textbf{L}, poprzez ustawianie wartości L[j, i] na iloraz U[j, i] i U[i, i], a od \textbf{U[i][i:]} odejmuje iloczyn \textbf{L[j][i]} i \textbf{U[i][i:]}. 
    \subsubsection{Pseudokod}
    \begin{lstlisting}[]
    function lu_decomposition(A):
        n = size(A) 
        L = identity_matrix(n)
        U = A
    
        for i = 0 to n-1:
            for j = i+1 to n-1:
                L[j][i] = U[j][i] / U[i][i]
                for k = i to n-1:
                    U[j][k] = U[j][k] - L[j][i] * U[i][k]
    
        return L, U
    \end{lstlisting}

    \subsubsection{Implementacja w Pythonie}
    \begin{lstlisting}[language=Python]
    def lu_decomposition(A):
        n = len(A)
        L = np.eye(n)
        U = A.copy()
    
        for i in range(n):
            for j in range(i + 1, n):
                L[j, i] = U[j, i] / U[i, i]
                U[j, i:] -= L[j, i] * U[i, i:]
    
        return L, U
    \end{lstlisting}
    \subsubsection{Analiza złożoności}
    Eliminacja wartości z pierwszej kolumny wymaga $n$ operacji dodawania i $n$ operacji mnożenia dla $n-1$ elementów. Mamy zatem $2n(n-1)$ operacji dla pierwszej kolumny. Dla drugiej kolumny wykonujemy $2(n-i)(n-i+1)$ operacji. Łączna ilość operacji do wykonania wynosi zatem:

    \begin{equation}
        \sum_{i}^{n}2(n-i)(n-i+1)
    \end{equation}

    \begin{equation}
        2\sum_{i}^{n}(n-i)(n-i) + (n - 1)
    \end{equation}

    Poprzez analogię do równania 4, możemy wywnioskować, że przybliżona złożoność będzie wynosić $\frac{2n^3}{3}$. Złożoność obliczeniowa jest więc rzędu  \( O(n^3) \).

    
    \subsection{Algorytm LU faktoryzacji z pivotingiem}
    \subsubsection{Opis algorytmu}
    Faktoryzacja z pivotingiem stosuje inny sposób wyboru elementu głównego każdego wiersza. Analogicznie do Eliminacji Gaussa, wybieramy element o największej wartości bezwględnej z każdego wiersza każdej macierzy, a następnie zamieniamy wiersze macierzy tak, aby główne elementy znalazły się na przekątnej.  
    \subsubsection{Pseudokod}
    \begin{lstlisting}[]
    function lu_decomposition_pivoting(A):
    n = length(A)
        L = zeros(n, n)
        U = copy(A)
        P = eye(n)
    
        for i from 1 to n do:
            pivot_row = argmax(abs(U[i:, i])) + i
            if pivot_row != i then:
                swap rows U[i] and U[pivot_row]
                swap rows L[i] and L[pivot_row]
                swap rows P[i] and P[pivot_row]
    
            for j from i + 1 to n do:
                L[j, i] = U[j, i] / U[i, i]
                U[j, i:] -= L[j, i] * U[i, i:]
    
        L = eye(n) + L
        return transpose(P), L, U

    \end{lstlisting}
    \subsubsection{Implementacja w Pythonie}
    \begin{lstlisting}[language=Python]
    def lu_decomposition_pivoting(A):
        n = len(A)
        L = np.zeros((n, n))
        U = A.copy()
        P = np.eye(n)
    
        for i in range(n):
            pivot_row = np.argmax(np.abs(U[i:, i])) + i
            if pivot_row != i:
                U[[i, pivot_row]] = U[[pivot_row, i]]
                L[[i, pivot_row]] = L[[pivot_row, i]]
                P[[i, pivot_row]] = P[[pivot_row, i]]
    
            for j in range(i + 1, n):
                L[j, i] = U[j, i] / U[i, i]
                U[j, i:] -= L[j, i] * U[i, i:]
    
        L = np.eye(n) + L
        return P.T, L, U
    \end{lstlisting}
    \subsubsection{Analiza złożoności}
    W przypadku faktoryzacji LU pivoting również nie wpływa znacznie na rząd złożoności obliczeniowej, przez co pozostaje on równy \( O(n^3) \).
    \subsubsection{Porównanie z Scipy}

    \begin{lstlisting}[language=Python]
        L:
        [[1.         0.         0.        ]
        [2.51331659 1.         0.        ]
        [1.12521846 0.47304452 1.        ]]
        [[ 1.          0.          0.        ]
        [ 0.39788064  1.          0.        ]
        [ 0.44770263 -0.06369218  1.        ]]
        U:
        [[ 0.02773589  0.04715342  0.07249239]
        [ 0.         -0.07082049 -0.08996962]
        [ 0.          0.         -0.02421129]]
        [[ 0.06970907  0.04769098  0.09222671]
        [ 0.          0.0281781   0.03579717]
        [ 0.          0.         -0.02421129]]

    \end{lstlisting}
    
    Widzimy wyniki L i U zwróce przez naszą funkcję (bez pivotingu!) i przez funkcję
    biblioteczną `scipy.linalg.lu`. WIdzimy, że wyiki się od siebie różnią. Wynika to z tego, że
    funkcja biblioteczna stosuje pivoting. Jeśli użyejmy naszej funkcji z pivotingiem to uzyskamy
    identyczne wyniki. Warto zauważyć, że nawet w wersji bez pivotingu, jeśli trafi się nam
    "szczęśliwa" macierz, to macierz permutacji będzie równa macierzy identycznościowej
    i wtedy oba algorytmu: nasz bez pivotingu i biblioteczny zwrócą identyczny wynik.
    Dla n = 3 takie prawdopodobieństwo jest równe $1/6$


    
\end{document}