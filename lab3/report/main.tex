\documentclass[12pt,a4paper,table]{article}

\usepackage[a4paper,
            tmargin=2cm,
            bmargin=2cm,
            lmargin=2cm,
            rmargin=2cm,
            bindingoffset=0cm]{geometry}

\usepackage{lmodern}
\usepackage[T1]{polski}
\usepackage[utf8]{inputenc}
\usepackage{tocloft}
\usepackage{hyperref}
\usepackage{amsmath}
\usepackage{listings}
\usepackage{graphicx}
\usepackage{subfig}
\usepackage{float}
\usepackage{booktabs}

\hypersetup{
    colorlinks,
    citecolor=black,
    filecolor=black,
    linkcolor=black,
    urlcolor=black
}

\newtheorem{definition}{Def}


\begin{document}
    \title {
        Rachunek Macierzowy i Statystyka Wielowymiarowa \\
        Sprawozdanie 3 \\
        Normy Macierzowe i SVD

    }

    \author{
        Adam Staniszewski \\
        Przemysław Węglik
    }

    \date{\today}

    \maketitle

    \tableofcontents
    \newpage

    \section{Norma M$_1$}
    
    \subsection{Obliczanie normy}
    
    \begin{lstlisting}[language=Python]
    matrix_1_norm = lambda x: np.max(np.sum(x, axis=0))
    \end{lstlisting}

    Każdą z norm obliczamy dla macierzy 

    \[
    \begin{matrix}
    4 & 9 & 2 \\
    3 & 5 & 7 \\
    8 & 1 & 6
    \end{matrix}
    \]

    zwaną dalej macierzą M.

    \vspace{1em}
    M$_1$ dla macierzy M wynosi 15.
    
    \subsection{Obliczanie współczynnika uwarunkowania}
    
    \begin{lstlisting}[language=Python]
    cond_matrix_1_norm = lambda x: matrix_1_norm(x) * 
                            matrix_1_norm(np.linalg.inv(x))
    \end{lstlisting}

    \vspace{1em}
    Współczynnik uwarunkowania macierzy M wynosi w tym przypadku 1.0000000000000004.
    
    \section{Norma M$_2$}
    
    \subsection{Obliczanie normy}

    \begin{lstlisting}[language=Python]
    matrix_2_norm = lambda x: np.linalg.svd(M, compute_uv=False)[0]
    \end{lstlisting}

    \vspace{1em}
    M$_2$ dla macierzy M wynosi 15.000000000000002.
    
    \subsection{Obliczanie współczynnika uwarunkowania}

    \begin{lstlisting}[language=Python]
    def cond_matrix_2_norm(matrix):
        singular_values = np.linalg.svd(matrix, compute_uv=False)
        return singular_values[0] / singular_values[matrix.shape[0] - 1]
    \end{lstlisting}

    \vspace{1em}
    Współczynnik uwarunkowania macierzy M wynosi w tym przypadku 4.330127018922198.
    
    \section{Norma M$_p$}
    
    Obliczenie normy M$_p$ jest problemem NP-Trudnym. Odsyłamy do publikacji \href{https://arxiv.org/pdf/1001.2613.pdf}{Approximating Matrix p-norms}.

    
    \section{Norma M$_\infty$}
    
    \subsection{Obliczanie normy}

    \begin{lstlisting}[language=Python]
    matrix_inf_norm = lambda x: np.max(np.sum(x, axis=1))
    \end{lstlisting}

    M$_\infty$ dla macierzy M wynosi 15.
    
    \subsection{Obliczanie współczynnika uwarunkowania}

    \begin{lstlisting}[language=Python]
    cond_matrix_inf_norm = lambda x: matrix_inf_norm(x) * 
                            matrix_inf_norm(np.linalg.inv(x))
    \end{lstlisting}

    \vspace{1em}
    Współczynnik uwarunkowania macierzy M wynosi w tym przypadku 1.0000000000000002.
    
    \section{SVD}
    Używamy algorytmy \textbf{Power Iteration}.

    \begin{lstlisting}[language=Python]
    def power_iteration(A, num_iterations: int):
        # Ideally choose a random vector
        # To decrease the chance that our vector
        # Is orthogonal to the eigenvector
        b_k = np.random.rand(A.shape[1])
    
        for _ in range(num_iterations):
            # calculate the matrix-by-vector product Ab
            b_k1 = np.dot(A, b_k)
    
            # calculate the norm
            b_k1_norm = np.linalg.norm(b_k1)
    
            # re normalize the vector
            b_k = b_k1 / b_k1_norm
    
        # Rayleigh quotient
        return b_k, b_k.T @ A @ b_k / (b_k.T @ b_k)
    \end{lstlisting}
    
\end{document}
